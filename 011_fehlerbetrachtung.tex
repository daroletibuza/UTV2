\chapter{Fehlerbetrachtung}
\label{sec:fehler}
In diesem Abschnitt erfolgt die Fehlerbetrachtung des Versuches, welche Einfluss auf die Messergebnisse haben können.\\
Die Bestimmungen via Sinnesprüfung erfolgten sehr subjektiv. Zwar wurde die Meinung über Geruch, Färbung und Trübung zwischen allen drei Praktikumsteilnehmern diskutiert, es sollten die Ergebnisse in Tab. \ref{tab:einstufungen} und Tab. \ref{tab:einstufung_geruch} dennoch als kritisch betrachtet werden und der Fokus eher auf die qualitativen Messergebnisse gerichtet werden.\\
Für die elektrochemische Messung wurden die Proben 1 bis 3 sowohl filtriert und unfiltriert untersucht. Die Messwerte in Tab. \ref{tab:elektrochemisch} unterscheiden sich dabei nur minimal. Die Unterschiede in Leitfähigkeit und pH-Wert können sich dabei auf die gefilterten Schwebstoffe zurückführen lassen. Sind minimale Stoffkonzentrationen an ihnen adsorbiert, könnten diese schon Einfluss auf den pH-Wert nehmen.\\
Deutlich bedeutsamer für die Fehlerbetrachtung scheint in diesem Versuch jedoch die pH-Wert-Elektrode. Sie ist laut dem Messgerät schon zu alt (dargestellt durch ein blinkendes Elektrodensymbol) und hat dementsprechend lange Ansprechzeiten, um Veränderungen des pH-Wertes aufzunehmen. Diese müsste dementsprechend gewechselt werden, um zuverlässigere Ergebnisse zu erhalten. Zudem ist auch die Kalibrierung der pH-Elektrode zu hinterfragen. Diese ist auf \SI{25}{\celsius} eingestellt, obwohl im Schnitt Raumtemperaturen zwischen 17,5-\SI{18}{\celsius} gemessen wurden. Da der pH-Wert Temperaturabhängig ist, würden sich durch die falsche Kalibrierung Abweichungen zum tatsächlichen pH-Wert feststellen lassen. Diese Abweichungen betragen bei dieser Temperaturdifferenz von \SI{5}{\kelvin} jedoch nur $\approx 1\%$ und sind für die grobe Einschätzung in diesem Versuch vernachlässigbar \cite{WasserPH}. \\
Im Weiteren erfolgt die Fehlerbetrachtung der foto- bzw. kolorimetrischen Messdaten (siehe Tab. \ref{tab:fotometrisch}).
\textit{Die Betrachtung des gesamten anorganisch gebundenen Stickstoffs oder Phosphors ist nur sinnvoll, wenn das Wasser praktisch frei von Biomasse ist. Die Biomasse würde ansonsten später zersetzt werden und dann die enthaltenen Nährstoffe abgeben, die aber nicht erfasst wurden, da sie sich ja nicht in Lösung befanden. Die Proben eins und zwei enthielten nicht zersetzte Biomasse und unterliegen damit obigem Einfluss.} \\
Das Testen der Wasserproben mit Schnellteststäbchen ist eine schnelle aber weniger genaue Analysemethode. Der Vergleich des eingefärbten Indikators mit dem Verpackungsaufdruck ist zum Teil an die subjektive Farbwahrnehmung des Testers gebunden und gibt daher nur eine grobe Einschätzung. Die Auswertung der Indikatorfärbung mit dem Reflektometer umgeht diesen Einflussfaktor. Jedoch können auch hier durch zufällige Messfehler und durch mangelhafte Kalibrierung Abweichungen vom realen Wert auftreten. Die Indikatoren unterliegen vermutlich auch diversen anderen Umwelteinflüssen, welche das Farbresultat abweichen lassen. %So befanden sich die Probentemperatur und der pH-Wert bei den durchgeführten Messungen zwar immer im erlaubten Bereich, aber
Die Analyse mit Schnellteststreifen erforderte die Einhaltung gewisser Einwirkzeiten. Die Einwirkzeiten wichen gelegentlich um wenige Sekunden vom empfohlenen Wert ab.
\newpage
Das Alter der Schnellteststreifen ist ein weiterer entscheidender Einflussfaktor. Einige der Teststreifen hatten ihr Mindesthaltbarkeitsdatum bereits um mehr als ein Jahr überschritten. Die genauen Auswirkungen der Überalterung sind nicht bekannt.
Betroffen waren davon die Reflektometertestreagenzien für Ammonium, die Schnellteststreifen für Nitrat und die Schnellteststreifen für Phosphat.\\
Zudem erfolgte die Verdünnung der Probe unter erschwerten Bedingungen, da der anfänglich verwendete Messzylinder einen großen Riss aufwies, durch welchen ein Teil der Lösung verloren ging. Das unverzügliche Umfüllen der verbliebenen Lösung ermöglichte die Weiterverwendung der Probe. Es wurde angenommen, dass die ausgetretene Flüssigkeit bereits ideal durchmischt war und so die Konzentration nicht beeinflusst wurde.  

