\chapter{Fehlerbetrachtung}
\label{sec:fehler}

Die Betrachtung des gesamten anorganisch gebundenen Stickstoffs oder Phosphors ist nur sinnvoll, wenn das Wasser praktisch frei von Biomasse ist. Die Biomasse würde ansonsten später zersetzt werden und dann die enthaltenen Nährstoffe abgeben, die aber nicht erfasst wurden da sie sich ja nicht in Lösung befanden. Die Proben eins und zwei enthielten nicht zersetzte Biomasse und unterliegen damit obigem Einfluss. \\
Das testen der Wasserproben mit Schnellteststäbchen ist eine schnelle aber weniger genaue Analysemethode. Der Vergleich des eingefärbten Indikators mit dem Verpackungsaufdruck ist zum Teil an die subjektive Farbwahrnehmung des Testers gebunden. Die Auswertung der Indikatorfärbung mit dem Reflektometer umgeht diesen Einflussfaktor. Jedoch können auch hier durch zufällige Messfehler und durch mangelhafte Kalibrierung Abweichungen vom realen Wert auftreten. Die Indikatoren unterliegen vermutlich auch diversen anderen Umwelteinflüssen welche das Farbresultat abweichen lassen. %So befanden sich die Probentemperatur und der pH-Wert bei den durchgeführten Messungen zwar immer im erlaubten Bereich, aber

Die Analyse mit Schnellteststreifen erforderte die Einhaltung gewisser Einwirkzeiten.Die Einwirkzeiten wichen gelegentlich um wenige Sekunden vom empfohlenen Wert ab.

Das Alter der Schnellteststreifen ist ein weiterer entscheidender Einflussfaktor. Einige der Teststreifen hatten ihr Mindesthaltbarkeitsdatum bereits um mehr als ein Jahr überschritten. Die genauen Auswirkungen der Überalterung sind nicht bekannt.
Betroffen waren davon die Reflektometertestreagenzien für Ammonium, die Schnellteststreifen für Nitrat und die Schnellteststreifen für Phosphat

Die Verdünnung der Probe eins erfolgte unter erschwerten Bedingungen da der anfänglich verwendete Messzylinder einen großen Riss aufwies durch welchen ein Teil der Lösung verloren ging. Das unverzügliche umfüllen der verbliebenen Lösung ermöglichte die Weiterverwendung der Probe. Es wurde angenommen, dass die ausgetretene Flüssigkeit bereits ideal durchmischt war und so die Konzentration nicht beeinflusst wurde.  

