\chapter{Ergebnisse}
\label{sec:ergebnisse}

Im folgenden Protokollabschnitt werden die Versuchsergebnisse der Versuchsdurchführung präsentiert.\\

Die aus Versuchsteil 1 wahrgenommenen Einstufungen der Abwasserproben 1 bis 3 sind in Tab. \ref{tab:optisch_sp} aufgeführt und im Vergleich mit Abb. \ref{fig:proben} nachzuvollziehen.
\vspace*{-2.5mm}
\renewcommand{\arraystretch}{1.2}
\begin{table}[h!]
	\centering
	\caption{Wahrgenommene Einstufungen der Färbung und Trübung der Abwasserproben 1 bis 3}
	\label{tab:optisch_sp}
	%\resizebox{10cm}{!}{
	\begin{tabulary}{\textwidth}{C|CCC}
		\hline
		\textbf{} & \textbf{Probe 1} & \textbf{Probe 2} & \textbf{Probe 3} \\ 
		\hline
		\textbf{Färbung} & stark gefärbt& stark gefärbt& schwach \\
		\textbf{Trübung} & undurchsichtig & undurchsichtig& klar \\
		\hline
	\end{tabulary}
	%}
\end{table}
\FloatBarrier
\vspace*{-2.5mm}

%Tabelle Ende

Für die Abwasser Proben 1 bis 3 sind die wahrgenommenen Einschätzungen zu den Gerüchen in Tab. \ref{tab:olfaktorisch_sp} aufgeführt.

\vspace*{-2.5mm}
\renewcommand{\arraystretch}{1.2}
\begin{table}[h!]
	\centering
	\caption{Beschreibung des Geruchs der Abwasserproben 1 bis 3}
	\label{tab:olfaktorisch_sp}
	%\resizebox{10cm}{!}{
	\begin{tabulary}{\textwidth}{C|CCC}
		\hline
		\textbf{} & \textbf{Probe 1} &  \textbf{Probe 2}&  \textbf{Probe 3}\\
		\hline
		\textbf{Intensität}	& schwach	&stark &ohne\\
		\textbf{Art}		& erdig, muffig	& modrig	&-\\
		\textbf{charak.-chemisch  }	& - & - & -\\
		\hline
	\end{tabulary}
	%}
\end{table}
\FloatBarrier
\vspace*{-0.5mm}

%Tabelle Ende

Die elektrochemischen Messwerte für die Abwasserproben 1 bis 3 sind in Tab. \ref{tab:elektrochemisch} eingetragen.
Zu beachten ist dabei, dass die Messwerte für den pH-Wert nur eine grobe Einschätzung liefern, da laut Messgerät die pH-Wert-Elektrode hätte gewechselt werden müssen. \\
Weitere Diskussionen zur Fehlerbetrachtung sind unter Abschnitt \ref{sec:fehler} zu finden.

\vspace*{-2.5mm}
\renewcommand{\arraystretch}{1.2}
\begin{table}[h!]
	\centering
	\caption{Elektrochemische Messwerte der Abwasserproben 1 bis 3}
	\label{tab:elektrochemisch}
	%\resizebox{10cm}{!}{
	\begin{tabulary}{\textwidth}{L|C|C|C}
		\hline
		\textbf{} 						& \textbf{pH-Wert\protect\footnotemark[1] $\boldsymbol{\left[-\right]}$} & \textbf{Leitfähigkeit $\boldsymbol{\left[\si{\milli \siemens \per \centi \meter}\right]}$}& \textbf{Temperatur} $\boldsymbol{\left[\si{\celsius}\right]}$\\
		\hline
		\textbf{Probe 1}				& 1,70	& 10,59	& 17,5	\\
		\textbf{Probe 1 (filtr.)}		& 1,68  & 10,32	& 17,6	\\
		\hline
		\textbf{Probe 2}				& 7,38 	& 2,67	& 17,4	\\
		\textbf{Probe 2 (filtr.)}		& 7,58	& 2,55	& 17,8	\\
		\hline
		\textbf{Probe 3}				& 7,44	& 2,43	& 17,4	\\
		\textbf{Probe 3 (filtr.)}		& 7,54	& 2,27	& 17,9	\\
		\hline
	\end{tabulary}
	%}
\end{table}
\FloatBarrier
\vspace*{-2.5mm}

%Tabelle Ende


\footnotetext[1]{pH-Wert-Elektrode auf \SI{25}{\celsius} kalibriert}


\newpage 

Die für die fotometrischen bzw. kolorimetrischen Messungen erhobenen Daten lassen sich für die Abwasserproben 1 bis 3 in Tab. \ref{tab:fotometrisch} wiederfinden. In dieser Tabelle sind sowohl die Schnelltestergebnisse (ST) als auch die Reflektometerergebnisse (RM) aufgeführt. \\
Ebenfalls aufgeführt ist die Berechnung der Konzentration an Nitrat \ce{NO3-} mittels Schnelltest für Probe 1, da der Messbereich von maximalen \SI{500}{\milli \gram \per \liter} nicht ausreicht. Die dafür nötige Berechnung der Verdünnung der Probe 1 ist ab Gleichung \ref{gl1} gezeigt.

\vspace*{-2.5mm}
\renewcommand{\arraystretch}{1.2}
\begin{table}[h!]
	\centering
	\caption{Elektrochemische Messwerte der Abwasserproben 1 bis 3}
	\label{tab:fotometrisch}
	%\resizebox{10cm}{!}{
	\begin{tabulary}{1.05\textwidth}{L|C|C|C}
		\hline
		\textbf{} 						& \textbf{Probe 1} $\boldsymbol{\left[\si{\milli \gram \per \liter}\right]}$& \textbf{Probe 2}$\boldsymbol{\left[\si{\milli \gram \per \liter}\right]}$&\textbf{Probe 3}$\boldsymbol{\left[\si{\milli \gram \per \liter}\right]}$\\
		\hline
		\textbf{Nitrit \ce{NO2-} (Schnelltest)}	& 2	& 0&2\\
		\textbf{Nitrit \ce{NO2-} (Reflektometer)} & <0,3  & <0,3	&<0,3\\
		\hline
		\textbf{Nitrat \ce{NO3-} (ST)}			& 500 {\footnotesize oder mehr} ?	& 0	&0\\
		\textbf{Nitrat \ce{NO3-} (RM)}			& 48 ?	& <3	&7,4\\
		\hline
		\textbf{Phosphat \ce{PO4^3-} (ST)}		& {\footnotesize kein Messwert}\protect\footnotemark[2]	& 0	&25\\
		\textbf{Phosphat \ce{PO4^3-} (RM)}		& {\footnotesize kein Messwert}\protect\footnotemark[2]	& 12	&27\\
		\hline
		\textbf{Ammonium \ce{NH4+} (RM)}		& {\footnotesize kein Messwert}\protect\footnotemark[2]		&	1,3	&0,6\\
		\hline
	\end{tabulary}
	%}
\end{table}
\FloatBarrier
\vspace*{-2.5mm}

%Tabelle Ende

\footnotetext[2]{keine Messung möglich, da laut Packungsbeilage  der pH-Wert zu niedrig für ein verwertbares Ergebnis ist}

\subsubsection{\underline{Verdünnung für Nitrat-Gehaltbestimmung der Probe 1}}
\textbf{Es wird von $\boldsymbol{V_{\text{Probe 1}} = \SI{100}{\milli \liter}}$ ausgegangen:}
\begin{flalign}
\label{gl1}
	\frac{\SI{500}{\milli \gram \per \liter}}{\SI{50}{\milli \gram \per \liter}} &= \frac{\SI{100}{\milli \liter}}{V_{\text{zu verdünnen}}}\\[2mm]
	V_{\text{zu verdünnen}}	&= \SI{100}{\milli \liter}*\frac{\SI{50}{\milli \gram\per \liter}}{\SI{500}{\milli \gram\per \liter}}\\[-2mm]
			&=\underline{ \SI{10}{\milli \liter}} \\[3mm]
			V_{\ce{H2O}}	&= \SI{100}{\milli \liter}-V_{\text{zu verdünnen}}\\
					&=\SI{100}{\milli \liter}-\SI{10}{\milli \liter}\\
					&= \underline{\SI{90}{\milli \liter}}\\[3mm]
	\frac{\SI{10}{\milli \liter}}{\SI{100}{\milli \liter}}	&=\frac{1}{10}\\
					&=\underline{\underline{1:10}} \quad \text{{\footnotesize (Verdünnung)}}
\end{flalign}

Mit \SI{50}{\milli \gram \per \liter} ist der Messbereich des Schnellteststreifens für Nitrat mit \SI{0}{}-\SI{500}{\milli \gram \per \liter} gut erfasst und das Volumen von \SI{10}{\milli \liter} ist einfach mittels Messzylinder abzumessen.\\
Daraus folgt eine Verdünnung von 1:10 mit einem Volumenteil (\SI{10}{\milli \liter}) Abwasserprobe 1 und neun Volumenteilen (\SI{90}{\milli \liter}) destilliertes Wasser.\\

\newpage

MESSWERT FÜR 1:10 VERDÜNNUNG FEHLT