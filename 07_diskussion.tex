\chapter{Diskussion}
\label{sec:diskussion}
Die Wasserprobe 1 ist sehr trüb und riecht etwas erdig, muffig. Die starke Trübung und die dunkelbraune bis schwarze Farbe ist wahrscheinlich organischen Ursprungs. Es könnte sich hierbei um Wasser aus einem Sumpfgebiet handeln. Es kann kein Wasser sein, dass bereits durch den Boden gefiltert wurde.
Die für Moore typische anaerobe Zersetzung führt zu einem absinken des pH-Wertes. Der gleiche Prozess könnte auch in einem Faulturm stattgefunden haben. Tatsächlich wurde ein, mit rund 1,7, sehr niedriger pH-Wert für diese Probe festgestellt. Der niedrige pH-Wert verhinderte darauf folgende Phosphat- und Ammoniummessungen.\\

Die Wasserprobe 2 ist ebenfalls sehr dunkel gefärbt und sehr trüb. Noch dazu riecht sie stark moderig und faulig. Es handelt sich hierbei wahrscheinlich um ein Abwasser, in welchem organische Anteile von Mikroorganismen unter Ausstoß von Schwefelwasserstoff und anderen charakteristischen Geruchsstoffen zersetzt werden oder wurden. Auch dieses Wasser muss von der Oberfläche stammen. Die Probe enthält sehr wenig Ammonium- als auch Phosphat- und Nitrationen. Diese müssen noch in der Biomasse gebunden vorliegen.\\

Die Wasserprobe 3 ist nur schwach getrübt und verströmt einen sehr schwachen Geruch. Es sind vergleichsweise hohe Konzentrationen an Phosphat- und Nitrationen nachweisbar. Gegenüber der Probe 2 liegen weniger Ammoniumionen vor. Es liegt nah, dass die Probe 3 bereits einen Klärprozess durchlaufen hat, bei dem organische Trübstoffe abgebaut und die enthaltenen Nährstoffe freigesetzt wurden.\\

In Anbetracht dessen, dass die Proben im Rahmen eins Praktikums gegeben sind, wäre es möglich, dass sie aus verschieden Phasen der Abwasserbehandlung stammen. In diesem Fall könnte die Probe 2 das ankommende Schmutzwasser, die Probe 2 das Wasser während der Belüftung und die Probe 3 Wasser vor der Phosphatabscheidung darstellen. \\

Das Hauptaugenmerk der Untersuchungen liegt auf Phosphat und Stickstoffverbindungen, weil diese wichtige und in vielen Fällen limitierende Nährstoffe für das Pflanzenwachstum darstellen. Ihre Konzentration bestimmt das Algenwachstum in den Gewässern in besonderem Maße und ist damit für das Umkippen von aquatischen Ökosystemen verantwortlich. Um die Gewässer vor diesem Schicksal zu bewahren, müssen die Abwässer von obigen Nährstoffen befreit werden.\\



Die Mindestanforderungen für das Einleiten kommunaler Abwässer in den Vorfluter bei einer Anlage der Größenklasse 5 sind bei den getesteten Wasserproben nur zum Teil erfüllt, wie es aus der Tabelle \ref{tab_vgl} leicht zu lesen ist. Keine der drei Probe überschreitet den Grenzwert für den Ammoniumgehalt. Die Konzentration des anorganisch gebundenen Gesamtstickstoffs wird durch die Probe 1 um mehr als das achtfache überschritten. Dieses Abwasser müsste zur Einhaltung des Grenzwertes weiter denitrifiziert werden. Die Proben zwei und drei halten den Grenzwert für anorganisch gebundenen Gesamtstickstoff ein. Den Abwässern der Proben zwei und drei muss Phosphor entzogen werden, da sie den Grenzwert von \SI{1}{\milli\gram\per\liter} beinahe um das vier- und achtfache überschreiten, während in der Probe eins keine Belastung mit Phosphor festgestellt werden konnte.
Die Abbildung \ref{Balkendiagramm} visualisiert obige Zusammenhänge nochmals anschaulich.


Die Abbildungen \ref{ph} bis \ref{pges} stellen die gemessenen Parameter der Abwasserproben im Vergleich zu üblichen Häuslichen Abwässern dar. Aus der Abbildung \ref{ph} geht hervor, dass die Proben zwei und drei mittelstark belastet sind. Der etwas basische pH-Wert ließe sich zum Beispiel sehr kostengünstig durch die Zugabe kleiner Mengen einer Säure korrigieren. Die Einstufung der Belastungsgrade anhand des p-H-Wertes in der Art wie es hier erfolgen soll, weist eine entscheidende Schwäche auf. Abwässer welche sehr Sauer sind gelten demnach als gering belastet. Ein Abwasser mit dem p-H-Wert von 2 kann in natürlichen Gewässern und an der Abwasserentsorgungsinfrastruktur bei großen Einleitvolumina mit Sicherheit beträchtlichen Schaden anrichten. 

Die Gesamtstickstoffgehalte der Abwässer werden in der Abbildung \ref{nges} dargestellt. Es geht daraus hervor, dass die Probe 1 stark belastet ist. Die Proben zwei und drei sind im Vergleich zu Häuslichen Abwässern sehr gering belastet.

Die Abbildung \ref{pges} stellt bildet schließlich die Werte für den Gesamtstickstoffgehalt im Vergleich zum Häuslichen Abwasser ab. Alle drei untersuchten Proben schneiden hier unterdurchschnittlich (gut) ab.

Die Behandlung der untersuchten Abwässer kann durch eine herkömmliche Kläranlage erfolgen. Beim Einleiten in die Kanalisation vermischen sich die Abwässer mit dem Abwasserstrom. Darum werden eventuell etwas überdurchschnittlich belastete Abwässer wie die Probe 1 verdünnt und stellen bei Ankunft im Klärwerk keine zusätzliche Herausforderung mehr dar, so lange der Anteil an der Gesamtabwassermenge vernachlässigbar bleibt.

