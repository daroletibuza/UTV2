\chapter{Diskussion}
\label{sec:diskussion}
Die Wasserprobe 1 ist sehr trüb und riecht etwas erdig, muffig. Die starke Trübung und die dunkelbraune bis schwarze Farbe ist wahrscheinlich organischen Ursprungs. Es könnte sich hierbei um Wasser aus einem Sumpfgebiet handeln. Es kann kein Wasser sein, dass bereits durch den Boden gefiltert wurde.
Die für Moore typische anaerobe Zersetzung führt zu einem absinken des pH-Wertes. Der gleiche Prozess könnte auch in einem Faulturm stattgefunden haben. Tatsächlich wurde ein, mit rund 1,7, sehr niedriger pH-Wert für diese Probe festgestellt. Der niedrige pH-Wert verhinderte darauf folgende Phosphat- und Ammoniummessungen.\\

Die Wasserprobe 2 ist ebenfalls sehr dunkel gefärbt und sehr trüb. Noch dazu riecht sie stark moderig und faulig. Es handelt sich hierbei wahrscheinlich um ein Abwasser, in welchem organische Anteile von Mikroorganismen unter Ausstoß von Schwefelwasserstoff und anderen charakteristischen Geruchsstoffen zersetzt werden oder wurden. Auch dieses Wasser muss von der Oberfläche stammen. Die Probe enthält sehr wenig Ammonium- als auch Phosphat- und Nitrationen. Diese müssen noch in der Biomasse gebunden vorliegen.\\

Die Wasserprobe 3 ist nur schwach getrübt und verströmt einen sehr schwachen Geruch. Es sind vergleichsweise hohe Konzentrationen an Phosphat- und Nitrationen nachweisbar. Gegenüber der Probe 2 liegen weniger Ammoniumionen vor. Es liegt nah, dass die Probe 3 bereits einen Klärprozess durchlaufen hat, bei dem organische Trübstoffe abgebaut und die enthaltenen Nährstoffe freigesetzt wurden.\\

In Anbetracht dessen, dass die Proben im Rahmen eins Praktikums gegeben sind, wäre es möglich, dass sie aus verschieden Phasen der Abwasserbehandlung stammen. In diesem Fall könnte die Probe 2 das ankommende Schmutzwasser, die Probe 2 das Wasser während der Belüftung und die Probe 3 Wasser vor der Phosphatabscheidung darstellen. \\

Das Hauptaugenmerk der Untersuchungen liegt auf Phosphat und Stickstoffverbindungen, weil diese wichtige und in vielen Fällen limitierende Nährstoffe für das Pflanzenwachstum darstellen. Ihre Konzentration bestimmt das Algenwachstum in den Gewässern in besonderem Maße und ist damit für das Umkippen von aquatischen Ökosystemen verantwortlich. Um die Gewässer vor diesem Schicksal zu bewahren, müssen die Abwässer von obigen Nährstoffen befreit werden.\\

anorganisch gebundener Gesamtstickstoff und phosphor

Die Betrachtung des gesamten anorganisch gebundenen Stickstoffs oder Phosphors ist nur sinnvoll, wenn das Wasser praktisch frei von Biomasse ist. Die Biomasse würde ansonsten später zersetzt werden und dann die enthaltenen Nährstoffe abgeben, die aber nicht erfasst wurden da sie sich ja nicht in Lösung befanden.
Der gesamte anorganisch gebundene Stickstoff kann über den Masseanteil des Elementaren Stickstoffs an den einfachen anorganischen Verbindungen Nitrat-, Nitrit- und Ammoniumionen berechnet werden.\\
Dabei kann aus dem Massenanteil des Stickstoffs am Molekül abgeleitet werden, dass $\SI{1}{\milli\gram\per\liter}$Ammoniumionen $\frac{7}{9}\si{\milli\gram\per\liter}$ anorganisch gebundenem Stickstoff entsprechen.(vgl. Gleichung 5.1)
\begin{flalign}
	\frac{M(N)}{M(NH_4)} &= \frac{\SI{14,007}{\gram\per\mole}}{\SI{14,007}{\gram\per\mole}+4*\SI{1,008}{\gram\per\mole}} =\frac{7}{9}\si{\milli\gram\per\liter}\\
	\frac{M(N)}{M(NO_3)} &= \frac{\SI{14,007}{\gram\per\mole}}{\SI{14,007}{\gram\per\mole}+3*\SI{15,999}{\gram\per\mole}} =\frac{7}{9}\si{\milli\gram\per\liter}\\
	\frac{M(N)}{M(NO_2)} &= \frac{\SI{14,007}{\gram\per\mole}}{\SI{14,007}{\gram\per\mole}+2*\SI{15,999}{\gram\per\mole}} =\frac{7}{9}\si{\milli\gram\per\liter}
\end{flalign}

Zur Berechnung des anorganisch gebundenen Gesamtstickstoffs können die Ionenkonzentrationen mit dem entsprechenden Umrechnungsfaktoren aufsummiert werden.


Gegenüberstellung von elektro fotometrischen und kolorimetrisch
tabellarisch und Graphisch gegenüberstellen und diuskutieren

Diagramme aus Abfall???

Einstufen im Vergleich zu häuslichen Abwässern und Behandlungsempfehlung geben

