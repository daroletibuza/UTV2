\chapter{Diskussion}
\label{sec:diskussion}
In diesem Abschnitt des Protokolls werden nun die Ergebnisse des Abschnittes \ref{sec:ergebnisse} diskutiert und ausgewertet.\\\\
Begonnen wird mit der Auswertung der Sinnesprüfung und der Einschätzung über die Herkunft der Abwasserproben 1 bis 3.\\ 
Die Wasserprobe 1 ist sehr trüb und riecht leicht erdig und muffig. Die starke Trübung und die dunkelbraune bis schwarze Farbe ist wahrscheinlich organischen Ursprungs. Es könnte sich hierbei um Wasser aus einem Sumpfgebiet handeln. Es kann kein Wasser sein, dass bereits durch den Boden gefiltert wurde.
Die für Moore typische anaerobe Zersetzung führt zu einem absinken des pH-Wertes. Der gleiche Prozess könnte auch in einem Faulturm stattgefunden haben. Tatsächlich wurde ein sehr niedriger pH-Wert für diese Probe von 1,7 festgestellt (siehe Tab \ref{tab:elektrochemisch}). Der niedrige pH-Wert verhinderte darauffolgende Phosphat- und Ammoniummessungen mittels foto- bzw. kolorimetrischer Messmethoden.\\ \\
Die Wasserprobe 2 ist ebenfalls sehr dunkel gefärbt und sehr trüb. Noch dazu riecht sie stark moderig und faulig. Es handelt sich hierbei wahrscheinlich um ein Abwasser, in welchem organische Anteile von Mikroorganismen unter Ausstoß von Schwefelwasserstoff und anderen charakteristischen Geruchsstoffen zersetzt werden oder wurden. Auch dieses Wasser wird höchstwahrscheinlich von der Oberfläche stammen. Die Probe enthält im Vergleich sehr wenig Ammonium- als auch Phosphat- und Nitrationen (siehe Tab. \ref{tab:elektrochemisch}). \textit{Diese müssen noch in der Biomasse gebunden vorliegen.}\\ \\
Die Wasserprobe 3 ist nur schwach getrübt und es ist nur ein sehr schwacher Geruch wahrnehmbar. Es sind vergleichsweise hohe Konzentrationen an Phosphat- und Nitrationen nachweisbar (siehe Tab. \ref{tab:elektrochemisch}). Gegenüber der Probe 2 liegen weniger Ammoniumionen vor. Es liegt nah, dass die Probe 3 bereits einen Klärprozess durchlaufen hat, bei dem organische Trübstoffe abgebaut und die enthaltenen Nährstoffe freigesetzt wurden sein könnten.\\ \\
In Anbetracht dessen, dass die Proben im Rahmen eines Praktikums gegeben sind, wäre es möglich, dass sie aus verschieden Phasen der Abwasserbehandlung stammen. In diesem Fall könnte die Probe 2 das ankommende Schmutzwasser, die Probe 2 das Wasser während der Belüftung und die Probe 3 Wasser vor der Phosphatabscheidung darstellen. \\
\newpage
Das Hauptaugenmerk der Untersuchungen liegt auf Phosphat- und Stickstoffverbindungen, da diese wichtige und in vielen Fällen limitierende Nährstoffe für das Pflanzenwachstum darstellen.
Stickstoffverbindungen gelangen dabei als Abbauprodukte organischer Verbindungen wie Harnstoffe oder Fäkalien, sowie aus industriellen Abwässern an. Gerade die Umwandlung von Ammonium- in Nitratverbindungen verbraucht Sauerstoff, der beispielsweise dem Ökosystem See fehlt und zum Fischsterben führen kann. Zudem verschärft sich dieser Sauerstoffverbrauch immer weiter, da die entstehenden Nitrate als Dünger für Pflanzen wirken und bei deren Absterben ebenfalls wieder Sauerstoff für die Zersetzung nötig ist \cite{NundP}. Des Weiteren kann Ammoniak gelöst bereits in geringen Konzentrationen zu einem Sterben von Wasserlebewesen führen \cite{AnalyseAbwasserN}. \linebreak
Phosphorverbindungen wirken ähnlich sauerstoffzehrend wie Nitratverbindungen durch ihren Düngungseffekt bei Pflanzen. Diese Verbindungen stammen meist aus anorganischen Düngemitteln, welche in der Landwirtschaft eingesetzt werden und über Versickerung in das Grundwasser und somit in weitere Gewässer gelangt \cite{NundP,AnalyseAbwasserP}. \linebreak
Somit bestimmen die Konzentrationen der genannten Verbindungen das Algen- bzw. Pflanzenwachstum in den Gewässern in besonderem Maße und sind damit für das Umkippen von aquatischen Ökosystemen verantwortlich. Um die Gewässer vor solch einer sogenannten Eutrophierung zu bewahren, müssen die Abwässer von obigen Nährstoffen befreit werden.\\\\
Im Weiteren folgt die Diskussion ob die elektrochemischen, fotometrischen und kolorimetrischen Bestimmungen der Abwasserproben den Mindestanforderungen für das Einleiten kommunaler Abwässer in einen Vorfluter der Größenklasse 5 entsprechen \cite[S. 29]{Skript}.\\
Die Mindestanforderungen für das Einleiten kommunaler Abwässer in den Vorfluter bei einer Anlage der Größenklasse 5 sind bei den getesteten Wasserproben 1 bis 3 nur teilweise erfüllt, wie in Tab. \ref{tab_vgl} und Abb. \ref{Balkendiagramm} nachzuvollziehen. Zu erkennen ist, dass keine der drei Proben den Grenzwert für den Ammoniumgehalt von \SI{10}{\milli \gram \per \liter} überschreitet. Stark auffallend ist dafür jedoch die Konzentration des anorganisch gebundenen Gesamtstickstoffs, welche durch die Probe 1 um mehr als das achtfache überschritten wird. Dieses Abwasser müsste zur Einhaltung des Grenzwertes weiter denitrifiziert werden. Die Proben 2 und 3 halten den Grenzwert für anorganisch gebundenen Gesamtstickstoff ein. Den Abwässern der Proben 2 und 3 muss Phosphor entzogen werden, da sie den Grenzwert von \SI{1}{\milli\gram\per\liter} beinahe um das vier- bzw. achtfache überschreiten. Für Probe 1 ist aufgrund des zu geringen pH-Wertes keine Bestimmung des Phosphatgehaltes mit gegebenen Mitteln möglich. Eine Belastung der Probe 1 mit Phosphaten ist daher nicht auszuschließen und müsste über alternative Messmethoden bestimmt werden.\\
Die Abbildungen \ref{ph} bis \ref{pges} stellen die gemessenen Parameter der Abwasserproben im Vergleich zu üblichen häuslichen Abwässern dar \cite[S. 29]{Skript}. Aus der Abbildung \ref{ph} in Bezug auf den pH-Wert geht hervor, dass die Proben 2 und 3 nahezu mittelstark belastet sind. Der etwas basische pH-Wert ließe sich zum Beispiel sehr kostengünstig durch die Zugabe kleiner Mengen einer verträglichen Säure korrigieren. Die Einstufung der Belastungsgrade anhand des pH-Wertes in der Art wie es hier erfolgen soll, weist eine entscheidende Schwäche auf:\linebreak
Abwässer welche sehr sauer sind, gelten nach Tab. \ref{tab:komm} als gering belastet. Ein Abwasser mit dem pH-Wert von Probe 1 kann in natürlichen Gewässern und an der Abwasserentsorgungsinfrastruktur bei großen Einleitvolumina beträchtlichen Schaden anrichten, da bereits ab einem pH-Wert von unter 5,5 Kleinstlebewesen geschädigt werden \cite{pHAbwasser}.\\
Die Gesamtstickstoffgehalte der Abwässer werden in der Abbildung \ref{nges} dargestellt. Es geht daraus hervor, dass die Probe 1 stark belastet ist. Die Proben 2 und 3 sind im Vergleich zu häuslichen Abwässern sehr gering belastet und erfordern keinen weiteren Handlungsbedarf.\\
Die Abbildung \ref{pges} bildet schließlich die Werte für den Gesamtphosphorgehalt im Vergleich zum häuslichen Abwasser ab. Alle drei untersuchten Proben schneiden in diesem Fall gut, sprich mit einer geringen Belastung, ab.\\

Die Behandlung der untersuchten Abwässer kann durch eine herkömmliche Kläranlage erfolgen. Beim Einleiten in die Kanalisation vermischen sich die Abwässer mit dem Abwasserstrom. Darum könnte man vermuten, dass zum Teil stark belastete Abwässer, wie die Probe 1, verdünnt werden würden und stellen bei Ankunft im Klärwerk keine zusätzliche Herausforderung mehr dar. Der Anteil an der Gesamtabwassermenge würde dabei vernachlässigbar bleiben.

